\documentclass{article}
\usepackage{hyperref}
\newcommand{\code}{{\texttt{QuickFiber}}}
\title{Documentation for \code\ Input and Output Files}
\author{Jaime E. Forero-Romero}

\begin{document}
\maketitle
\tableofcontents 
y
\vspace{1cm}
This documents describes the format of the files generated by the
\code\ software.  

\section{Important definitions}
Taken from \url{https://desi.lbl.gov/trac/wiki/Pipeline/FormatsAndNumbering}
\begin{itemize}

  \item Tile: pre-defined locations on the sky. Tiles are pre-defined
    with a 6-digit ID. 900000 and above are reserved for calibration,
    commissioning, test, ancillary, and other non-DESI key project
    tiles, which may not even appear in a DESI tile list.  
  \item Pointing: A specific
    selection of targets within a tile indicating both a pointing of the
    telescope and the fiber positioners.  
\end{itemize}

This means that a tile may be observed
multiple times with different pointings. A pointing may be observed
multiple times with different exposures. If even one target changes
or any target: fiber mapping changes it becomes a new pointing. A
slightly different positioner location of the same targets on the
same fibers is the same pointing.  

\section{\code\ inputs}
The necessary inputs of \code\ are 
\begin{itemize}
\item TargetDB. Database containing the information of all targets on the sky. 
\item ObsDB. Database constructed from the ongoing DESI
observations after the data has been processed by the Spectroscopic
pipeline. This DB has not been designed yet. 
\item SurveyTiles. File containing the positions of all the tiles to
  be observed. 
Currently this ASCII file\\
  \url{https://desi.lbl.gov/trac/browser/code/survey/surveyplan/trunk/data/desi-tiles-full.par}  
\item FiberPositions. File containing the positions of all fibers on
  the focal plane.  
Currently this ASCII file\\
  \url{https://desi.lbl.gov/trac/browser/code/desimodel/trunk/data/focalplane/fiberpos.txt} 
\item PlateScale. File containing the mapping between the radius from
  the center of the focal plane and the radial angle.
  Currently a fit from the data obtained in this ASCII file\\
  \url{https://desi.lbl.gov/trac/browser/code/desimodel/trunk/data/focalplane/platescale.txt} 
\end{itemize}



\section{\code\ inputs data-structure}


The table of interest in the TargetDB \texttt{/project/projectdirs/desi/db}
is \texttt{Target} with the following columns

\begin{itemize}
\item id: primary key
\item cand\_ID [0-]
\item ra: degrees [0-360]
\item dec: degrees [-90 -+90]
\item priority: [0-] Larger positive numbers indicate higher priority.
\item nobs: [0-] Desired number of spectroscopic observations in
  DESI.
\item objtype: ELG, LRG, QSO, SKY, STDSTAR, GAL, OTHER
\end{itemize}

\noindent
Columns of interest to \code\ from ObsDB will be

\begin{itemize}
\item cand\_ID [0-]
\item ra: degrees [0-360]
\item dec: degrees [-90 -+90]
\item fiber\_ID [0:]
\item positioner\_ID  [0:]
\item spectrograph\_ID [0:]
\item tile\_ID [1:28810]
\item signaltonoise
\item confirmedobjtype: ELG, LRG, QSO, SKY, STDSTAR, GAL, OTHER
\end{itemize}

\noindent
SurveyTiles is an ASCII file including the following fields
\begin{itemize}
\item tile\_ID [1:28810]
\item ra: degree [0-360]
\item dec: degrees [-90-+90]
\item layer [1-5]
\item in\_desi [0,1]
\end{itemize}

\noindent
FiberPositions is an ASCII file including the following fields.
\begin{itemize}
\item fiber\_ID [0:]
\item positioner\_ID  [0:]
\item spectrograph\_ID [0:]
\item x: mm, position on focal plane
\item y: mm, position on focal plane
\item z: mm, position on focal plane
\end{itemize}

\noindent
PlateScale is an ASCII file including the following fields.
\begin{itemize}
\item Radius: mm. Radius from the center of the focal plane.
\item Theta: degrees. Radial angle.
\end{itemize}

\section{\code\ outputs}

The principal outputs of \code\ are
\begin{itemize}
\item FiberMap. A pointing defining which targets are on which
  fiber.
\item PotentialFiberMap. The set of targets that can be reached by a
  set of fibers in a  tile.  
\end{itemize}

The previous two sets will be stored in the same FITS file for each
tile and pointing. This means that two different assignment algorithms
running on the same tile producing different pointings will store
the results in different files. 

\subsection{File naming convention}

Primary output files from \code\ will follow the naming scheme

\begin{center}
\code\_{\texttt{TTTTTT}}\_{\texttt{PPPPPPPPPPPP.fits}}
\end{center}
\noindent
where \texttt{TTTTTT} is the 6-digit Tile ID and \texttt{PPPPPPPPPPPP} is
a 12-digit pointing ID. The pointing ID is hash value computed by
adding the IDs of all the targets in the pointing and taking the last
12 digits. 

\subsection{File type and structure}
The outputs are uncompressed FITS files with all relevant
information about the code in the primary HDU header, a BIN table
describing the pointing, a BIN table listing the potential targets. 

\begin{center}
\begin{tabular}{|l|l|l|}\hline
HDU0 & NULL & Empty\\\hline
HDU1 & Binary FITS table & FiberMap information\\\hline
HDU2 & Binary FITS table & PotentialFiberMap information\\\hline
\end{tabular}
\end{center}

\subsection{Data-structure}

FiberMap is a table with the following columns.
\begin{itemize}
\item fiber: [0-4999]
\item positioner: [0-4999]
\item numtarget: [0-] Number of potential target IDs associated to each
  fiber. 
\item objtype: ELG, LRG, QSO, SKY, STDSTAR, GAL, OTHER. 
\item targetid: 
  unique target identifier to get back to target
  selection info. Corresponds to the cand\_ID in the TargetDB.
\item desi\_target0: 64 bit mask of targeting info. 
\item ra: degrees [0-360]. 
\item dec: degrees [-90 - +90]. 
\item xfocal\_design: mm from center in positioner coordinate system 
\item yfocal\_deisgn: mm from center in positioner coordinate system  
\end{itemize}

\noindent
PotentialFiberMap with only one column.
\begin{itemize}
\item potentialtargetid:  unique target identifier to get back to target
  selection info. This contains  all the targets that can be
  reached by a fiber. Corresponds to the cand\_ID in the TargetDB. 
  This concatenates the sets of available
  targets for each fiber.
\end{itemize}
All the arrays have the same size as the number of fibers, except
potentialtargetid which has a variable size.



\subsection{Example of potentialtargetid construction}
Let's assume for simplicity that we have 4 fibers with IDs
\texttt{[0,1,2,3]}. There are in turn 4 different sets of targets that
can be reached by each fiber. These sets are \texttt{[980,203]},
\texttt{[736,102,304]}, \texttt{[0]} and \texttt{[234]} for fibers
\texttt{0,1,2} and \texttt{3}, respectiveley.
The index \texttt{-1} for fiber \texttt{3} means that this fiber
cannot reach any target.

In this case fiber, numtarget and potentialtargetid are as follows.
\begin{itemize}
\item
fiber: contains the array \texttt{[0,1,2,3]}.
\item
numtarget: contains the array \texttt{[2,3,0,1]}.
\item
potentialtargetid: contains the array \texttt{[980,203,736,102,304,234]}.

\end{itemize}





\end{document}
